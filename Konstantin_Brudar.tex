% LaTeX resume using res.cls
\documentclass[line,margin]{resume} 
%\usepackage{helvetica} % uses helvetica postscript font (download helvetica.sty)
%\usepackage{newcent}   % uses new century schoolbook postscript font 

\usepackage[utf8]{inputenc}
\usepackage[T2A]{fontenc}
\usepackage[english,russian]{babel}
\usepackage[colorlinks]{hyperref}

\begin{document}

\name{Константин Брударь, 22 года (22.04.1993)}
% \address used twice to have two lines of address
\address{г. Москва, м. Авиамоторная}
\address{+7 926 701-67-94, brudar@bk.ru}

 
\begin{resume}
 
\section{ЦЕЛЬ}       
    Работа, стажировка в сфере разработки программного обеспечения 
 
 
\section{ОБРАЗОВАНИЕ}
    {\sl Бакалавр,} прикладная математика и информатика \hfill июнь 2016 \\
    % \sl will be bold italic in New Century Schoolbook (or
    % any postscript font) and just slanted in
    % Computer Modern (default) font
    МГУ имени М. В. Ломоносова \\
    Факультет вычислительной математики и кибернетики \\
    Кафедра суперкомпьютеров и квантовой информатики
 
 
\section{НАВЫКИ}
    {\sl Программирование:}
        Python, C, C++, Bash, Makefile \\
%    {\sl Параллельное программирование:} OpenMP, MPI, CUDA \\
    {\sl Операционные системы:}
        Windows, Linux \\
    {\sl Библиотеки, верстка:}
        Numpy, Matplotlib, QuTiP, \LaTeX, HTML, CSS
 
\section{ОПЫТ \\ РАБОТЫ}
    {\sl Программист} \hfill сентябрь--декабрь 2015 \\
        Производственная практика в вузе
        \begin{itemize}  \itemsep -2pt % reduce space between items
            \item
                Разработка программы для моделирования динамики квантовой системы
            \item
                Проектирование сайта кафедры 
        \end{itemize}

    {\sl Специалист по программному обеспечению} \hfill июнь--сентябрь 2015 \\
        ООО СовИнТех, \href{www.svnth.ru}{www.svnth.ru} \\
        Производственный департамент Парус
        \begin{itemize}  \itemsep -2pt % reduce space between items
            \item
                Настройка программной системы Парус 
            \item
                Консультирование клиентов по работе с программой 
        \end{itemize}

 
\section{СЕРТИФИКАТЫ}
    \begin{itemize}  \itemsep -2pt % reduce space between items
        \item
            Программирование на Python \\
            \href{https://stepic.org/certificate/dc3bc62e168ed5459a4e654541fb4e40e4ba52ac.pdf}{https://stepic.org/certificate/dc3bc62e168ed5459a4e654541fb4e40e4ba52ac.pdf}
        \item
            Алгоритмы и структуры данных \\
            \href{https://stepic.org/certificate/e5a183e2fd2058b6c8e3e39c9ee04347869389b2.pdf}{https://stepic.org/certificate/e5a183e2fd2058b6c8e3e39c9ee04347869389b2.pdf}
    \end{itemize}


\section{ЯЗЫКИ}
    Русский --- {\it родной} \\
    Английский --- {\it читаю профессиональную литературу} \\


\end{resume}
\end{document}
